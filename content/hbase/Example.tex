%!TEX root = ../dokumentation.tex

\section{Example Project}
\label{lblHBaseExample}

This chapter's intention is to illustrate some of the technical facts discussed earlier. We used a standalone HBase setup on a virtual machine running Linux Ubuntu. For details on the installation process visit \href{www.hbase.apache.org/book.html\#quickstart}{www.hbase.apache.org/book.html\#quickstart}.

The following examples are executed with the HBase Shell, which is included in HBase's binaries. To start the shell execute 
\begin{lstlisting}[caption={start HBase shell},label=lst:hbas_shell]
	$ ./bin/hbase shell
\end{lstlisting}
in the HBase directory. 

\textbf{Create table student} \\
When creating a new table in HBase you need to specifiy the table's name as well as at least one column family. Additionally you can add parameter to specify the table setup. 
\begin{lstlisting}[caption={create table},label=lst:create_table]
	> create 'student', {NAME=>'personal',VERSIONS=>5}, 'school'
\end{lstlisting}
This command creates the table named \textit{student} with two column families \textit{personal} and \textit{school}. The column family personal is configured to store five versions of each column cell. This means that overwritten values are accessible until five overwrites. 

\textbf{Add row} \\
To insert a value you need to provide table name, row key, column family, column qualifier and value. 

\begin{lstlisting}[caption={insert values},label=lst:insert_values]
	> put 'student', 'Alexandra', 'personal:street', 'Schillerstrasse 31'
	> put 'student', 'Alexandra', 'personal:hair', 'blond'
\end{lstlisting}

The first line adds the value \textit{Schillerstraße 31} to the row \textit{Alexandra} in the table \textit{student}. The column family of this value is \textit{personal} and the column qualifier is \textit{street}. 

\textbf{Scan table} \\
When executing the following command all table entries are listed.
\begin{lstlisting}[caption={scan table},label=lst:scan_table]
	> scan 'student'
\end{lstlisting}

The output will look similar to this:

\begin{lstlisting}[caption={output of the table scan 1},label=lst:table_scan_output]
	ROW          COLUMN+CELL
	Alexandra    column=personal:hair, timestamp=1490775709059, value=blond
	Alexandra    column=personal:street, timestamp=1490775705439, value=Schillerstrasse 31
	1 row(s) in 0.0600 seconds
\end{lstlisting}

\textbf{Update value} \\
The value \textit{blond} should be updated to \textit{pink}.
\begin{lstlisting}[caption={update value},label=lst:update_value]
	> put 'student', 'Alexandra', 'personal:hair', 'pink'
\end{lstlisting}
Therfor again tablename, rowkey, column family and column qualifier are needed.
The output of scan table will look like this:
\begin{lstlisting}[caption={output of the table scan 2},label=lst:table_scan_output_2]
	ROW          COLUMN+CELL
	Alexandra    column=personal:hair, timestamp=1490776188067, value=pink
	Alexandra    column=personal:street, timestamp=1490775705439, value=Schillerstrasse 31
	1 row(s) in 0.0370 seconds
\end{lstlisting}

Only the most recent value is shown. To get more versions of an updated value a different command is needed.

\begin{lstlisting}[caption={get multiple versions},label=lst:get_multiple_versions]
	> get 'student', 'Alexandra', {COLUMN=>'personal',VERSIONS=>2}
\end{lstlisting}

Now two versions of the values stored within column family \textit{personal} are displayed.
\begin{lstlisting}[caption={output of the table scan 3},label=lst:table_scan_output_3]
	COLUMN                                       CELL
	personal:hair        timestamp=1490776188067, value=pink
	personal:hair        timestamp=1490775709059, value=blond
	personal:street      timestamp=1490775705439, value=Schillerstrasse 31
	3 row(s) in 0.0880 seconds
\end{lstlisting}
\textbf{Note:} This is only possible if the column family is set to more than one version on table initiation. 

\textbf{Types} \\
To demonstate stat HBase does not enforce types, we updated a string value to an integer value.
\begin{lstlisting}[caption={put integer into cell},label=lst:put_integer]
	> put 'student', 'Alexandra', 'personal:hair', 3
\end{lstlisting}
HBase executes this without any (error) message. In fact everything is just stored as byte array.
The output looks like this:
\begin{lstlisting}[caption={output of the table scan 4},label=lst:table_scan_output_4]
	ROW          COLUMN+CELL
	Alexandra    column=personal:hair, timestamp=1490777492998, value=3
	Alexandra    column=personal:street, timestamp=1490775705439, value=Schillerstrasse 31
	1 row(s) in 0.0170 seconds
\end{lstlisting}

\textbf{Drop table} \\
To delete the table it must be disabled before it can be dropped.
\begin{lstlisting}[caption={drop table},label=lst:drop_table]
	> disable 'student'
	> drop 'student'
\end{lstlisting}