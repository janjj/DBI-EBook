\section{Security}

Storing sensitive data today requires a high standard of IT security. If databases store personal data or business secrets it is necessary to protect these data against loss, destruction and theft. For this reason the  database architects  should be informed about the actual security mechanism, authentication mechanisms and about existing security gaps. This chapter describe the base authentication mechanisms, security issues and the administration of users.

\subsection{User Authentication}
After the first installation of CouchDB every user disposes automatically over privileges of an administrator. Every user can create database or change system properties. This very big security gap is also known as "Admin Party". Since version 0.11 it is possible to differ between three user roles \\ \cite{ApacheSoftwareFoundation.2013.SecurityFeatures}:
\begin{enumerate}
\item Server administrator \\
The user role of a server administrator disposes of all privileges. They have reading access and writing access to all databases stored on the CouchDB server and can change all server settings \cite{Anderson.2010.Buch}.
\item Database administrator

The database administrator disposes of reading access and writing access of one specific database. This user role have the privileges to create, edit, delete documents but "they can not create and neither delete [other] databases." \cite{ApacheSoftwareFoundation.2013.SecurityFeatures}
Additional they can create new database members and database admins for these specific database.
\item Database members \\
The database members have reading access to all documents but they have only writing access to 'normal documents'. Especially they can not change design documents.
\end{enumerate}
\textbf{Creating server administrator}\\
There are two technical capabilities to create a new server administrator. The first possibility is to change the \textit{local.ini} file \\ \cite{ApacheSoftwareFoundation.2013.AdminAccount}:
\begin{lstlisting}[frame=single, caption=Example Create new Server Administrator with local.in File \protect\cite{ApacheSoftwareFoundation.2013.AdminAccount}]
[admins]
username = password
\end{lstlisting}

Currently every admin has the possibility to read these plain-text password. This situation is a very big security gab. Only after restart CouchDB the password will be encoded with a hash. This hash will be create within three steps \cite{Anderson.2010.Buch}:
\begin{enumerate}
\item creates a new 128-bit [Universally Unique Identifier] (random numbers with low collision probability) = \textbf{salt} \cite{Anderson.2010.Buch}
\item "Creates a sha1 hash of the concatenation of the bytes of the plain-text password and the salt (sha1(password + textbf{salt}))" \\ \cite{Anderson.2010.Buch}
\item "Prefixes the result with -hashed- and appends textbf{salt}" \\ \cite{Anderson.2010.Buch}
\end{enumerate}
After restart CouchDB the \textit{local.ini} looks like:
\begin{lstlisting}[frame=single, caption=Example Create new Server Administrator with local.ini File \protect\cite{ApacheSoftwareFoundation.2013.AdminAccount}]
[admins]
username = -hashed-207b1b4f8434dc60429672c0c2ba3aae61568d6c,96406178a0718239acb72cb4e8f2e66e
\end{lstlisting}
But it also possible to create a server administrator with the provided API (for further information please see chapter 1.4.3 "The Database Part"): \\ \cite{Anderson.2010.Buch}
\begin{lstlisting}[frame=single, caption=Example Create new Server Administrator with REST API \protect\cite{Anderson.2010.Buch}]
curl -X PUT \$HOST/\_config/admins/username -d '"password"'
\end{lstlisting} 

Until now it is only possible to create a new databases after you entered the administrator password. \\
\\
\textbf{Creating database administrator and database members}
\\
To create a database administrator or a simple database member it is necessary to change the security object of the selected database. These object is located under \textit{\slash db\_name\slash\_security} \\ \cite{ApacheSoftwareFoundation.2013.SecurityFeatures}.
The data in these object is in JSON format \cite{ApacheSoftwareFoundation.2013.SecurityFeatures}:
\begin{lstlisting}[frame=single, caption=Example Create new Database Members and Database Administrator \protect\cite{ApacheSoftwareFoundation.2013.SecurityFeatures}]
{
  "admins" : {
     "names" : ["ines", "tobi"],
     "roles" : ["boss"]
   },
   "members" : {
     "names" : ["leon"],
     "roles" : ["producer", "consumer"]
   }
}
\end{lstlisting}

\textbf{If a database have no database users every user can change regular documents!}
CouchDB stores data of the single users in an user authentication database named (\_\textbackslash users). Every user have a separate document. In this document it is possible to save the properties of every user for example id, name or user roles.The property "type" can only substituted with the key "user". An example user definition in such a document is shown in the following graphic: \cite{ApacheSoftwareFoundation.2013.SecurityFeatures}:
\begin{lstlisting}[frame=single, caption=Example User Definition \protect\cite{ApacheSoftwareFoundation.2013.SecurityFeatures}]
{
 "\_id":"org.couchdb.user:ines",
 "type":"user",
 "name":"ines",
 "roles":["guest"],
 "password\_sha":"fe95df1ca59a9b567bdca5cbaf8412abd6e06121",
 "salt":"4e170ffeb6f34daecfd814dfb4001a73"
}
\end{lstlisting} 

\subsection{Cookie Authentication}
The standard user authentication with user name and password shows a increasingly security gap. Therefore, CouchDB offers the possibility of user authentication with Cookies. By the use of Cookies users can be authenticate without filling user-related data in permanent opened password dialogues. 
CouchDB offers the possibility to use Proxy or implement particular authentication modules to sync or authenticate user data with existing LDAP systems or Active Directory \cite{Anderson.2010.Buch}.

\subsection{Validation Function}
To be able to guarantee the consistency of the data in a database system, it is helpful to provide rules directives. Afterwards every document must fulfill these requirements. For example, it is possible to define directives which ensures that the field "name" may not bee empty. If this field will be empty an error message will be shown. \cite{Scheliga.2010}.

\begin{lstlisting}[frame=single, caption=Example Validation Function \protect\cite{Scheliga.2010}]
{
    \_id: "\_design/designdoc",
    validate\_doc\_update: "function(newDoc, oldDoc, userCtx) {
    if(newDoc.name === undefined) {
    throw {forbidden: 'Document must have a name.'};
    }}"
}
\end{lstlisting}

The deposited validation directives launch automatically if a new document created or an existing document changed. Every validation directives can include an arbitrarily number of these rules. However every design document can only deposit one validation function \cite{Scheliga.2010}.


